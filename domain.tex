% !TEX root=talk.tex
\section{Mathematics and Other Material}

%--------------------------------------------------------------------
%--- Mathematics
%--------------------------------------------------------------------

\begin{frame}[fragile]
\frametitle{Mathematics}
\begin{center}\begin{minipage}{.75\textwidth}\rmfamily
Equation \eqref{eq:gratio} relates the golden ratio and the Fibonacci series. 
Recall that the golden ratio is $\varphi = \frac{1}{2} (1 + \sqrt{5})$.

\begin{equation}\label{eq:gratio}
\varphi = 1 + \sum^{\infty}_{n=1}
                \frac{ (-1)^{n+1} }{ F_n F_{n+1} }
\end{equation}
\end{minipage}
\end{center}


\begin{center}\begin{minipage}{.9\textwidth}
\begin{beamerboxesrounded}{}
\vskip-1em
\begin{lstlisting}[escapechar=|,basicstyle=\ttfamily\small,moretexcs=eqref,emph={equation}]
Equation \eqref{eq:gratio} relates the golden ratio and the 
Fibonacci series. Recall that the golden ratio is |\textcolor{red}{\large\ttfamily\$}|\varphi =
\frac{1}{2} (1 + \sqrt{5})|\textcolor{red}{\large\ttfamily\$}|.

\begin{equation}\label{eq:gratio}
\varphi = 1 + \sum_{n=1}^{\infty}
                \frac{ (-1)^{n+1} }{ F_n F_{n+1} }
\end{equation}
\end{lstlisting}
\vspace*{-1em}
\end{beamerboxesrounded}
\end{minipage}
\end{center}

\end{frame}

%--------------------------------------------------------------------
%--- Program Listings
%--------------------------------------------------------------------

\begin{frame}[fragile]
\frametitle{Program Listings}

\begin{columns}
\begin{column}{.5\textwidth}
\begin{beamerboxesrounded}{}
\vspace{-1em}
\begin{lstlisting}[basicstyle=\ttfamily\footnotesize,escapechar=|,
emph={listings,lstlisting},moretexcs={color},commentstyle={},]
\usepackage{listings,xcolor}
...
\begin{lstlisting}[language=matlab,
basicstyle=\ttfamily,
keywordstyle=\bfseries\color{red},
commentstyle=\sffamily\color{green},
stringstyle=\rmfamily\color{orange}]
disp('Hello World');

% Calculate golden ratio
F = zeros(10, 1);
F(1:2) = [1, 1];
phi = 1; % Initial value  

for j=1:8
    F(j+2) = F(j+1) + F(j);
    phi = phi + (-1)^(j+1) ...
            / (F(j+1) * F(j));
end
|\bfseries\color{Maroon}\textbackslash end|{lstlisting}
\end{lstlisting}
\vspace{-1em}
\end{beamerboxesrounded}
\end{column}
\hfill\begin{column}{.5\textwidth}
\begin{lstlisting}[language=matlab,escapechar=~,lineskip=-2pt,
basicstyle=\ttfamily,
commentstyle=\upshape\sffamily\small\color{SeaGreen4},keepspaces=true,
keywordstyle=\bfseries\color{Maroon},stringstyle=\rmfamily\color{Sienna2}]
disp('Hello World');

% Calculate golden ratio
F = zeros(10, 1);
F(1:2) = [1, 1];
phi = 1; % Initial value

for j=1:8
    F(j+2) = F(j+1) + F(j);
    phi = phi + (-1)^(j+1) ...
            / (F(j+1) * F(j));
end
\end{lstlisting}
\end{column}
\end{columns}
\end{frame}

%--------------------------------------------------------------------
%--- Graphs and Diagrams
%--------------------------------------------------------------------

\begin{frame}[fragile]
\frametitle{Graphs and Diagrams}

\begin{columns}
\begin{column}{.65\textwidth}
\begin{beamerboxesrounded}{}
\vspace{-1em}
\begin{lstlisting}[basicstyle=\ttfamily,escapechar=|,
emph={xy,xymatrix},moretexcs={color},commentstyle={},]
\usepackage[all]{xy}

\[
  \xymatrix{
    A \ar[r]^f \ar[dr]_{f \circ g} &
    B \ar[d]^g \ar[dr]^{g \circ h} \\
    & C \ar[r]_h & D
  }
\]
\end{lstlisting}
\vspace{-1em}
\end{beamerboxesrounded}
\end{column}
\hfill\begin{column}{.35\textwidth}
{\Large
\[
\xymatrix{
A \ar[r]^f \ar[dr]_{f \circ g}
& B \ar[d]^g \ar[dr]^{g \circ h} \\
& C \ar[r]_h & D
 }
\]
}
\end{column}
\end{columns}
\end{frame}

%--------------------------------------------------------------------
%--- Tables
%--------------------------------------------------------------------

\begin{frame}[fragile]
\frametitle{Tables}
\begin{center}\rmfamily
\begin{tabular}{lrrr} \toprule
Year ending Mar 31 & 2016 & 2015 & 2014 \\ \midrule
Revenue & 14580.20 & 11900.40 & 8290.30 \\
Cost of sales & 6740.20 & 5650.10 & 4524.20 \\ \cmidrule(r){2-4}
\emph{Gross profit} & 7840.00 & 6250.30 & 3766.10 \\ 
\bottomrule
\end{tabular}
\end{center}

\begin{beamerboxesrounded}{}
\vskip-1em
\begin{lstlisting}[basicstyle=\ttfamily\small,
moretexcs={toprule,midrule,cmidrule,bottomrule,lightrulewidth},
alsoletter={23->}, emph={tabular,booktabs},
emph={[2]{b2-b3,>}}]
\usepackage{booktabs}
...
\begin{tabular}{lrrr} \toprule
  Year ending Mar 31 & 2016 & 2015 & 2014 \\ \midrule
  Revenue & 14580.20 & 11900.40 & 8290.30 \\
  Cost of sales & 6740.20 & 5650.10 & 4524.20 \\ \cmidrule(r){2-4}
  \emph{Gross profit} & 7840.00 & 6250.30 & 3766.10 \\ \bottomrule
\end{tabular}
\end{lstlisting}
\vspace{-1em}
\end{beamerboxesrounded}

\end{frame}

%--------------------------------------------------------------------
%--- Graph Plots
%--------------------------------------------------------------------

\begin{frame}[fragile]
\frametitle{Graph Plots}

\begin{columns}[totalwidth=\textwidth]
\begin{column}{0.6\textwidth}

% %\pgfplotsset{height=.75\textheight,width=\linewidth}
\begin{tikzpicture}[transform shape]
\begin{loglogaxis}[xlabel=DoF,
       width=\textwidth,
       height=0.6\textwidth]
\addplot table[x=dof,y=La] {examples/datafile.dat}; \addlegendentry{$L_\alpha$};
\addplot table[x=dof,y=Lb] {examples/datafile.dat}; \addlegendentry{$L_\beta$};
\end{loglogaxis} 
\end{tikzpicture}

\end{column}
\begin{column}{0.4\textwidth}

Contents of \texttt{datafile.dat}
\begin{beamerboxesrounded}{}
\vspace{-1em}
\lstinputlisting{examples/datafile.dat}
\vspace{-1em}
\end{beamerboxesrounded}

\end{column}
\end{columns}

%\medskip

\begin{beamerboxesrounded}{}
\vskip-1em
\begin{lstlisting}[basicstyle=\ttfamily\footnotesize,
emph={pgfplots,tikzpicture,loglogaxis},
moretexcs={addplot, table, addlegendentry,text},lineskip=-2pt]
\usepackage{pgfplots}
...
\begin{tikzpicture}
\begin{loglogaxis}[xlabel=DoF]
\addplot table[x=dof,y=La]{datafile.dat}; \addlegendentry{$L_\alpha$};
\addplot table[x=dof,y=Lb]{datafile.dat};  \addlegendentry{$L_\beta$};
\end{loglogaxis} 
\end{tikzpicture} 
\end{lstlisting}
\vspace{-1em}
\end{beamerboxesrounded}

\end{frame}

%--------------------------------------------------------------------
%--- Gantt Charts
%--------------------------------------------------------------------

\begin{frame}[fragile]
\frametitle{Gantt Charts}

\scalebox{.85}{\rmfamily
\begin{ganttchart}%
[y unit title=0.4cm,
y unit chart=0.5cm,
vgrid=true,
title/.style={draw=none, fill=RoyalBlue!50!black},
title label font=\sffamily\bfseries\color{white}, title label anchor/.style={below=-1.6ex},
title left shift=.05,
title right shift=-.05,
title height=1,
bar/.style={draw=none, fill=OliveGreen!75},
bar height=.6,
bar label font=\normalsize\color{black!50},
group right shift=0,
group top shift=.6,
group height=.3, group peaks height=.2,
bar incomplete/.style={fill=Maroon}]{1}{16} 
\gantttitle{2016}{4} \gantttitle{2017}{12} \\ 
\ganttbar%
[progress=100, bar progress label font=\small\color{OliveGreen!75}, progress label anchor/.style={right=4pt},
bar label font=\normalsize\color{OliveGreen},
name=pp]%
  {Preliminary Project}{1}{4} \\
\ganttset{progress label text={}, link/.style={black, -to}}
\ganttgroup{Objective 1}{5}{16} \\
\ganttbar[progress=50, name=T1A]{Task A}{5}{10} \\
\ganttlinkedbar[progress=0]{Task B}{11}{16} \\
\ganttgroup{Objective 2}{5}{16} \\
\ganttbar[progress=75, name=T2A]{Task A}{5}{13} \\
\ganttlinkedbar[progress=0]{Task B}{14}{16}
\end{ganttchart}
}

\begin{beamerboxesrounded}{}
\vskip-1em
\begin{lstlisting}[basicstyle=\ttfamily\footnotesize,lineskip=-2pt,
moretexcs={gantttitle,ganttbar,ganttlink,ganttgroup,ganttlinkedbar},
emph={pgfgantt,tikzpicture,ganttchart}]
\usepackage{pgfgantt}
...
\begin{ganttchart}[...settings...]{1}{16} 
\gantttitle{2016}{4} \gantttitle{2017}{12} \\ 
\ganttbar[progress=100]{Preliminary Project}{1}{4} \\ 
\ganttgroup{Objective 1}{5}{16} \\ 
\ganttbar[progress=50, name=T1A]{Task A}{5}{10} \\ 
\ganttlinkedbar[progress=0]{Task B}{11}{16} \\ 
...
\end{ganttchart} 
\end{lstlisting}
\vspace{-1em}
\end{beamerboxesrounded}

\end{frame}

%--------------------------------------------------------------------
%--- Chemistry
%--------------------------------------------------------------------

\begin{frame}[fragile]
\frametitle{Chemical Equations and Molecules}

\begin{center}\rmfamily\small
\ce{Zn^2+ <=>[\ce{+ 2OH-}][\ce{+ 2H+}]
$\underset{\text{amphoteres Hydroxid}}{\ce{Zn(OH)2 v}}$
<=>C[+2OH-][{+ 2H+}]
$\underset{\text{Hydroxozikat}}{\cf{[Zn(OH)4]^2-}}$
}
\hfil
\chemfig{H-C(-[2]H)(-[6]H)-C(-[7]H)=[1]O}
\end{center}

\begin{beamerboxesrounded}{}
\vskip-1em
\begin{lstlisting}[basicstyle=\ttfamily\small,moretexcs={ce,chemfig,underset,text,cf},
emph={mhchem,chemfig}]
\usepackage[version=3]{mhchem}   % sufficient for chemical equations
\usepackage{chemfig}             % for 2-D molecule drawings
...
\ce{Zn^2+ <=>[\ce{+ 2OH-}][\ce{+ 2H+}]
$\underset{\text{amphoteres Hydroxid}}{\ce{Zn(OH)2 v}}$
<=> C[+2OH-][{+ 2H+}] 
$\underset{\text{Hydroxozikat}}{\cf{[Zn(OH)4]^2-}}$ }

\chemfig{H-C(-[2]H)(-[6]H)-C(-[7]H)=[1]O}
\end{lstlisting}
\vspace*{-1em}
\end{beamerboxesrounded}

\end{frame}

%--------------------------------------------------------------------
%--- Circuits and SI Units
%--------------------------------------------------------------------

\begin{frame}[fragile]
\frametitle{Circuits and SI Units}
\begin{columns}
\begin{column}{.49\textwidth}\rmfamily
\begin{circuitikz}[transform shape,scale=.9]
\draw (0,0) node[anchor=east] {B}  to[short, o-*] (1,0)    to[R=20<\ohm>, *-*] (1,2)
  to[R=10<\ohm>, v=$v_x$] (3,2) -- (4,2)
  to[ cI=$\frac{\si{\siemens}}{5} v_x$, *-*] (4,0) -- (3,0)  to[R=5<\ohm>, *-*] (3,2)
  (3,0) -- (1,0)   (1,2) to[short, -o] (0,2) node[anchor=east]{A}
;\end{circuitikz}
\end{column}
\begin{column}{.49\textwidth}
\begin{itemize}\rmfamily
\item \SI{3.45d4}{\square\volt\cubic\lumen\per\farad}
\item \SIlist[per-mode=symbol]{40;85;103}{\kilo\metre\per\hour}
\end{itemize}
\end{column}
\end{columns}

\medskip

\begin{beamerboxesrounded}{}
\vskip-1em
\begin{lstlisting}[basicstyle=\ttfamily\footnotesize,lineskip=-2pt,
moretexcs={draw,si,siemens,ohm,SI,SIlist,square,volt,cubic,lumen,per,farad,kilo,metre,hour},
emph={siunitx,circuitikz},emph={[2]{draw,node,to}}
]
\usepackage{siunitx}
\usepackage[siunitx]{circuitikz}
...
\begin{circuitikz}
\draw (0,0) node[anchor=east] {B}
  to[short, o-*] (1,0)    to[R=20<\ohm>, *-*] (1,2)
  to[R=10<\ohm>, v=$v_x$] (3,2) -- (4,2)
  to[ cI=$\frac{\si{\siemens}}{5} v_x$, *-*] (4,0) -- (3,0)
  to[R=5<\ohm>, *-*] (3,2)
  (3,0) -- (1,0)   (1,2) to[short, -o] (0,2) node[anchor=east]{A}
;\end{circuitikz}

\SI{3.45d4}{\square\volt\cubic\lumen\per\farad}
\SIlist[per-mode=symbol]{40;85;103}{\kilo\metre\per\hour}
\end{lstlisting}
\vspace{-1em}
\end{beamerboxesrounded}
\end{frame}


\begin{frame}[fragile]
\frametitle{Bar Codes}
%%% CAUTION!!! This takes a LOOONG time to compile if you're using pdflatex. I'm just going to load the already generated PDFs instead.

% \begin{pspicture}
% \psbarcode{MECARD:N:Malaysia Open Source Conference 2011;TEL:+60196085482;URL:http://www.mosc.my/;EMAIL:secretariat@mosc.my;ADR:Bayview Beach Resort, Baru Ferringgi Penang;NOTE:Malaysia Open Source Conference 2011 (MOSC2011);;}{eclevel=L width=0.75 height=0.75}{qrcode}
% \end{pspicture}\;
\includegraphics[page=1]{images/talk-pics}\;
% \begin{pspicture}
% \psbarcode[scalex=0.7,scaley=0.7]{9781860742712}{ includetext guardwhitespace }{ean13} 
% \end{pspicture}\;
\includegraphics[page=2]{images/talk-pics}\;
% \begin{pspicture}
% \psbarcode[scalex=0.7,scaley=0.7]{978-3-86541-114}{includetext guardwhitespace}{isbn} 
% \end{pspicture}\;\;
\includegraphics[page=3]{images/talk-pics}\;
% \begin{pspicture}
% \psbarcode[scalex=0.7,scaley=0.7]{^453^178^121^239}{ columns=2 rows=10}{pdf417}
% \end{pspicture}%
\includegraphics[page=4]{images/talk-pics}
\llap{\raisebox{0.45in}{\includegraphics[page=5]{images/talk-pics}%
% \begin{pspicture}
% \psbarcode[scalex=0.6,scaley=0.6]{LE28HS9Z}{includetext}{royalmail}
% \end{pspicture}
}}

\bigskip

\begin{beamerboxesrounded}{}
\vskip-1em
\begin{lstlisting}[moretexcs={psbarcode},escapechar=|,basicstyle=\ttfamily\footnotesize,
emph={pst-barcode},emph={[2]{qrcode,ean13,isbn,pdf417,royalmail,}},
alsoletter={1347-}
]
\usepackage{auto-pst-pdf}  % Needed if running pdflatex; must use option -shell-escape
\usepackage{pstricks,pst-barcode}
...
|\color{Maroon}\bfseries\textbackslash begin|{pspicture}
\psbarcode{MECARD:N:Malaysia Open Source Conference...}{eclevel=L}{qrcode}
\psbarcode{9781860742712}{includetext guardwhitespace}{ean13} 
\psbarcode{978-3-86541-114}{includetext guardwhitespace}{isbn} 
\psbarcode{LE28HS9Z}{includetext}{royalmail}
\psbarcode{^453^178^121^239}{columns=2 rows=10}{pdf417}
|\color{Maroon}\bfseries\textbackslash end|{pspicture} 
\end{lstlisting}
\vspace{-1em}
\end{beamerboxesrounded}
\end{frame}

%--------------------------------------------------------------------
%--- Smart Diagrams
%--------------------------------------------------------------------

\begin{frame}[fragile]
\frametitle{`Smart Diagrams'}
\begin{columns}[T]

\begin{column}{.49\textwidth}
\resizebox{\linewidth}{!}{\smartdiagram[bubble diagram]{Planning Cycle,Assess,Plan,Implement,Renew}}

\begin{beamerboxesrounded}{}
\vskip-1em
\begin{lstlisting}[basicstyle=\ttfamily\footnotesize,lineskip=-2pt]
\usepackage{smartdiagram}
\smartdiagram[bubble diagram]{
  Planning Cycle,Assess,Plan,
  Implement,Renew}
\end{lstlisting}
\vspace{-1em}
\end{beamerboxesrounded}
\end{column}

\begin{column}{.49\textwidth}
\resizebox{\linewidth}{!}{\smartdiagram[priority descriptive diagram]{Assess,Plan,Implement,Renew}}

\begin{beamerboxesrounded}{}
\vskip-1em
\begin{lstlisting}[basicstyle=\ttfamily\footnotesize,lineskip=-2pt]
\usepackage{smartdiagram}
\smartdiagram
  [priority descriptive diagram]{
  Assess,Plan,Implement,Renew}
\end{lstlisting}
\vspace{-1em}
\end{beamerboxesrounded}
\end{column}

\end{columns}

% \resizebox{\linewidth}{\smartdiagram[priority descriptive diagram]{Planning Cycle,Assess,Plan,Implement,Renew}}
\end{frame}

%--------------------------------------------------------------------
%--- Crossword Puzzles
%--------------------------------------------------------------------

\begin{frame}[fragile]
\frametitle{Crossword Puzzles}
\begin{columns}
\begin{column}{.3\textwidth}\rmfamily
\begin{Puzzle}{5}{3}
|* |* |[1]E|X |* |.
|[2]A|[3]S|T |* |[4]T|.
|* |[5]P|A |R |T |.
\end{Puzzle}
\end{column}
\begin{column}{.65\textwidth}\rmfamily
\begin{PuzzleClues}{
\textbf{Across:} }
  \Clue{1}{EX}{unit of measure}
  \Clue{2}{AST}{\(\ast\)}
  \Clue{5}{PART}{sectioning unit}
\end{PuzzleClues}
\begin{PuzzleClues}{
\textbf{Down:} }
  \Clue{1}{ETA}{\(\eta\)}
  \Clue{3}{SP}{unit of measure}
  \Clue{4}{TT}{nonproportional font}
\end{PuzzleClues}
\end{column}
\end{columns}

\begin{beamerboxesrounded}{}
\vskip-1em
\begin{multicols}{2}
\begin{lstlisting}[escapechar=?,basicstyle=\ttfamily\footnotesize,
moretexcs={Clue},emph={cwpuzzle,Puzzle,PuzzleClues}]
\usepackage{cwpuzzle}
...
\begin{Puzzle}{5}{3}
|* |* |[1]E|X |* |.
|[2]A|[3]S|T |* |[4]T|.
|* |[5]P|A |R |T |.
\end{Puzzle}
\begin{PuzzleClues}{
\textbf{Across:} }
  \Clue{1}{EX}{unit of measure}
  \Clue{2}{AST}{\(\ast\)}
  \Clue{5}{PART}{sectioning unit}
\end{PuzzleClues}
\begin{PuzzleClues}{
\textbf{Down:} }
  \Clue{1}{ETA}{\(\eta\)}
  \Clue{3}{SP}{unit of measure}
  \Clue{4}{TT}{nonproportional font}
\end{PuzzleClues}
\end{lstlisting}
\end{multicols}
\vspace{-.5em}
\end{beamerboxesrounded}
\end{frame}

%--------------------------------------------------------------------
%--- Guitar Tabs
%--------------------------------------------------------------------

\begin{frame}[fragile]
\frametitle{Song Books with Guitar Tabs}
\vskip-.55\textheight
\begin{guitar}
\rmfamily\smallchords\def\chordsize{.5em}
\renewcommand\yoff{2}
\renewcommand\xoff{0}
\renewcommand\namefont{\footnotesize\sffamily}
\renewcommand\normalsiz{1.1}
\renewcommand\topfretsiz{1.2pt}
\newcommand{\CMaj}{\chord{t}{n,p3,p2,n,p1,n}{C}}
\newcommand{\Amin}{\chord{t}{n,n,p2,p2,p1,n}{Am}}
\newcommand{\FMaj}{\chord{t}{n,n,p3,p2,p1,p1}{F}}
\newcommand{\GMaj}{\chord{t}{p3,p2,n,n,n,p3}{G}}
Country [\CMaj]road, take me [\GMaj]home, to the [\Amin]place I be[\FMaj]long.
West Vir[\CMaj]ginia, mountain [\GMaj]momma, take me [\FMaj]home, country [\CMaj]road.
\end{guitar}

\begin{beamerboxesrounded}{}
\vskip-1em
\begin{lstlisting}[moretexcs={chord,CMaj,Amin,FMaj,GMaj},
emph={gchords,guitar},
basicstyle=\ttfamily\small]
\usepackage{gchords,guitar}
...
\begin{guitar}
\newcommand{\CMaj}{\chord{t}{n,p3,p2,n,p1,n}{C}}
\newcommand{\Amin}...
Country [\CMaj]road, take me [\GMaj]home, ...
\end{guitar}
\end{lstlisting}
\vspace{-1em}
\end{beamerboxesrounded}
\end{frame}

%--------------------------------------------------------------------
%--- Preview Environment
%--------------------------------------------------------------------

\begin{frame}[fragile]
\frametitle{Meh, What Good is That? Can't Use it Anywhere Else.}
Actually, you can.

\bigskip

\pause
\begin{beamerboxesrounded}{}
\vskip-1em
\begin{lstlisting}[moretexcs={PreviewEnvironment,texshade},basicstyle=\ttfamily,emph={preview}]
\usepackage[active,tightpage]{preview}
\PreviewEnvironment{texshade}
...
\begin{texshade}
...
\end{texshade}
\end{lstlisting}
\vspace{-1em}
\end{beamerboxesrounded}

\begin{itemize}
\item Run \texttt{pdflatex} $\rightarrow$ cropped \textsmaller{PDF} containing \emph{only} contents of \texttt{texshade}
\pause
\item ImageMagick: \verb|convert -depth 150 texshade.pdf texshade.png|
\pause
\item Multiple environments $\rightarrow$ multi-page \textsmaller{PDF} and multiple \textsmaller{PNG}s
\end{itemize}
\end{frame}

%--------------------------------------------------------------------
%--- Linguistics
%--------------------------------------------------------------------

% \begin{frame}[fragile]
% \frametitle{Linguistics}

% \begin{columns}[T]
% \begin{column}{.54\textwidth}\small\rmfamily
% \ex
% \begingl
% \gla \%*Wen liebt seine Mutter?//
% \glb Whom loves his mother//
% \glc `Who does his mother love?'//
% \endgl
% \xe

% \begin{beamerboxesrounded}{}
% \vskip-1em
% \begin{lstlisting}[moretexcs={ex,xe,gla,glb,glc},basicstyle=\ttfamily\footnotesize\lsstyle,
% emph={expex},
% lineskip=-2pt,commentstyle={}]
% \usepackage{linguex,qtree}
% ...
% \ex
% \begingl
% \gla \%*Wen liebt seine Mutter?//
% \glb Whom loves his mother//
% \glc `Who does his mother love?'//
% \endgl
% \xe
% \end{lstlisting}
% \vskip-1em
% \end{beamerboxesrounded}
% \end{column}
% \begin{column}{.45\textwidth}\footnotesize\rmfamily
% \Tree [ .S [.NP [.Pron He ] ] [.VP [.V kicked ] [.NP [.Det the ] [.N ball ] ] ] ]

% \medskip

% \begin{beamerboxesrounded}{}
% \vskip-1em
% \begin{lstlisting}[moretexcs={ex,xe,gla,glb,glcTree},basicstyle=\ttfamily\footnotesize\lsstyle,
% emph={expex,qtree},
% lineskip=-2pt,commentstyle={}]
% \usepackage{qtree}
% ...
% \Tree [ .S [.NP [.Pron He ] ] [.VP [.V kicked ] [.NP [.Det the ] [.N ball ] ] ] ]
% \end{lstlisting}
% \vspace*{-1em}
% \end{beamerboxesrounded}

% \end{column}
% \end{columns}

% \end{frame}

%--------------------------------------------------------------------
%--- Network Protocols
%--------------------------------------------------------------------

% \begin{frame}[fragile]
% \frametitle{Network Protocols}
% \begin{columns}
% \begin{column}{.505\textwidth}
% \begin{beamerboxesrounded}{}
% \vspace{-1em}
% \begin{lstlisting}[basicstyle=\ttfamily\footnotesize,
% moretexcs={bitheader,wordgroupr,bitbox,endwordgroupr,wordbox},
% emph={bytefield,rightwordgroup}]
% \usepackage{bytefield}
% ...
% \begin{bytefield}{16} 
% \bitheader{0,7,8,15} \\ 
% \begin{rightwordgroup}{Header} 
% \bitbox{4}{Tag} & \bitbox{12}{Mask} \\ 
% \bitbox{8}{Source} & 
% \bitbox{8}{Destination} 
% \end{rightwordgroup} \\ 
% \wordbox{3}{Data} 
% \end{bytefield} 
% \end{lstlisting}
% \vspace{-1em}
% \end{beamerboxesrounded}
% \end{column}
% \begin{column}{.49\textwidth}\rmfamily\small
% \hfill\begin{bytefield}[bitwidth=.75em]{16}
% \bitheader{0,7,8,15} \\ 
% \begin{rightwordgroup}{Header} 
% \bitbox{4}{Tag} & \bitbox{12}{Mask} \\ 
% \bitbox{8}{Source} & \bitbox{8}{Destination} 
% \end{rightwordgroup}\\ 
% \wordbox{3}{Data} 
% \end{bytefield} 
% \end{column}
% \end{columns}
% \end{frame}

%--------------------------------------------------------------------
%--- Life Sciences
%--------------------------------------------------------------------

% \begin{frame}[fragile]
% \frametitle{Life Sciences}

% \begin{texshade}{examples/AQPpro.MSF.txt}
% \shadingmode{similar} 
% \threshold[80]{50} 
% \setends{1}{80..112} 
% \hideconsensus 
% \feature{top}{1}{93..93}{fill:$\downarrow$}{first case (see text)} 
% \feature{bottom}{1}{98..98}{fill:$\uparrow$}{second case (see text)} 
% \end{texshade}
% \vskip-1em
% \begin{beamerboxesrounded}{}
% \vskip-1em
% \begin{lstlisting}[
% moretexcs={setends,shadingmode,threshold,hideconsensus,feature,downarrow,uparrow},
% emph={texshade},
% basicstyle=\ttfamily\small,lineskip=-2pt,escapechar=|]
% \usepackage{texshade}  % for nucleotide and peptide alignments
% ...
% \begin{texshade}{AQPpro.MSF.txt} 
% \shadingmode{similar} 
% \threshold[80]{50} 
% \setends{1}{80..112} 
% \hideconsensus 
% \feature{top}{1}{93..93}{fill:$\downarrow$}{first case (see text)} 
% \feature{bottom}{1}{98..98}{fill:$\uparrow$}{second case (see text)} 
% \end{texshade} 
% \end{lstlisting}
% \vspace{-1em}
% \end{beamerboxesrounded}
% \end{frame}

%--------------------------------------------------------------------
%--- Chess Games
%--------------------------------------------------------------------

% \begin{frame}[fragile]
% \frametitle{Chess games}

% \begin{columns}
% \begin{column}{.51\textwidth}

% \begin{beamerboxesrounded}{}
% \vskip-1em
% \begin{lstlisting}[basicstyle=\ttfamily\small,
% moretexcs={newgame,mainline,chessboard},
% emph={skak,chessboard}]
% \usepackage[skaknew]%
% {skak,chessboard}
% ...
% \newgame
% \mainline{1. e4 e5 2. Nf3 Nc6 3. Bb5 a6}
% \chessboard[smallboard]
% \end{lstlisting}
% \vspace{-1em}
% \end{beamerboxesrounded}
% \end{column}

% \begin{column}{.47\textwidth}
% \rmfamily
% \newgame\mainline{1. e4 e5 2. Nf3 Nc6 3. Bb5 a6}

% \chessboard[smallboard]

% \end{column}
% \end{columns}
% \end{frame}

%%% Local Variables:
%%% TeX-master: "talk"
%%% End:
