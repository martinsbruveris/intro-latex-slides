% Licence:
% This work is a derivative of "Multilingual 'Thank-You'" by
% LianTze Lim, used under a CC-BY 4.0 licence.
%
% This work is licenced under a CC-BY 4.0 licence by Martins Bruveris.
\documentclass{article}

\usepackage[active,tightpage]{preview}
\usepackage{varwidth}
\usepackage[dvipsnames,x11names,svgnames]{xcolor}
\usepackage{tikz}
\usepackage{polyglossia}
\usepackage{xeCJK} % For japanese font

\setotherlanguages{arabic,greek,russian}

% FreeSerif is VERY versatile when it comes to multilingual character sets.
% FreeSans is a little less so, so we'll have to re-define the sans-serif fonts for some languages later.
\setmainfont{FreeSerif}

\usetikzlibrary{positioning,backgrounds,chains,shapes}
\newcommand{\tqLemma}[4][]{%
\node[lemma] (#2) [draw=#3,fill=#3!30,
%label={[draw=#3!50,fill=#3!10]above:#2},
#1] {#4};
}

\tikzset{every node/.append style={draw}}
\tikzset{lemma/.append style={rectangle, 
                              rounded corners,
                              on chain}}
\tikzset{every label/.append style={semicircle, 
                                    font={\tiny},
                                    label distance=-.5pt,
                                    inner sep=1pt}}


\begin{document}

\begin{preview}
\begin{varwidth}{\linewidth}
\begin{tikzpicture}
\node (axis) [circle, inner sep=.3em, draw=gray, fill=gray!30] {};

% First layer (inner) circle
\begin{scope}[start chain=inner circle placed {at=(10+\tikzchaincount*32.72:6em)}] 
\tqLemma{ltn}{Orchid}{Gratias tibi}

\tqLemma{fra}{Rhodamine}{Merci}

\tqLemma{chn}{Salmon}{谢谢}

\tqLemma{rus}{SteelBlue}{\textrussian{Спасибо}}

\tqLemma{eng}{Goldenrod}{Thank you}

\tqLemma{grk}{CornflowerBlue}{Ευχαριστώ}

\tqLemma{lat}{Maroon}{Paldies}

\tqLemma{ara}{SeaGreen}{\textarabic{شكرا}}

\tqLemma{ger}{Teal}{Danke}

\tqLemma{glc}{Peach}{Tapadh leibh}

\newCJKfontfamily\jpfont{IPAPMincho}
\tqLemma[font=\jpfont]{jpn}{Crimson}{ありがとう}
\end{scope}

%% 2nd layer
% \begin{scope}[start chain=outer circle placed {at=(\tikzchaincount*32.72-16.36:10.5em)}] 
% \tqLemma{por}{Periwinkle}{Obrigado}
% \tqLemma{wel}{Maroon}{Diolch}
% \end{scope}

\begin{pgfonlayer}{background}
\path[draw=gray,thick,shorten >=-2pt] (axis) 
edge (ltn) 
edge (ger) 
edge (chn) 
edge (fra) 
edge (rus) 
edge (grk) 
edge (lat) 
edge (ara) 
edge (eng) 
edge (glc) 
edge (jpn);
\end{pgfonlayer}

\end{tikzpicture}
\end{varwidth}
\end{preview}
\end{document}

%%% Local Variables:
%%% TeX-master: t
%%% End:
